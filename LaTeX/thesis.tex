\documentclass[a4paper, 10pt, DIV=13, BCOR=8mm]{scrbook}

\usepackage[utf8]{inputenc}
\usepackage[T1]{fontenc}
\usepackage[ngerman]{babel}
\usepackage[a4paper, width=150mm, top=25mm, bottom=25mm, bindingoffset=6mm]{geometry}
\usepackage{
  , minted
  , fancyhdr
  , blindtext
  , todonotes
  , amsmath
  , amssymb
  , hyperref
}


\newcommand{\hs}[1]{\mintinline{haskell}{#1}}

\bibliographystyle{plain}


\begin{document}


\chapter*{Abstract}

\blindtext

\todo{Improve Abstract}


\tableofcontents


\chapter{Einleitung}

Man kann wie folgt eine wissenschaftliche Publikation zitieren \cite{christiansen2019verifying}.

\blindtext

Das folgende Beispiel zeigt, wie man Code-Beispiele in die Arbeit einbinden kann.
In diesem Fall wird ein Java-Klasse eingebunden.
\begin{minted}{java}
/*
  Hello World
 */
public class HelloWorld {
  public static void main(String[] args) {
    System.out.println("Hello World!");
  }
}
\end{minted}

\blindtext

Das folgende Beispiel zeigt, wie man gleichungsbasierte Umformungen in Haskell umsetzen kann.
\begin{align*}
\hs{and [False, True, True]} &= \hs{foldr (&&) True [False, True, True]}
\end{align*}

\blindtext

Das folgende Beispiel zeigt, wie man eine mathematische Funktion mit Fallunterscheidung definieren kann.
\begin{align*}
& !\colon \mathbb{N} \to \mathbb{N}\\
& n! = \begin{cases}
  1 & \mbox{falls}~n = 0\\
  n * (n-1)! & \mbox{sonst}
\end{cases}
\end{align*}

\blindtext


\bibliography{thesis}


\end{document}
